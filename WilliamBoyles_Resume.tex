%-------------------------------------------------------------------------------
% CONFIGURATIONS
%-------------------------------------------------------------------------------
% A4 paper size by default, use 'letterpaper' for US letter
\documentclass[a4paper]{awesome-cv}

% Configure page margins with geometry
\geometry{left=0.5in, top=0.25in, right=0.5in, bottom=0.25in, footskip=0.0in}

% Color for highlights
% Awesome Colors: awesome-emerald, awesome-skyblue, awesome-red, awesome-pink, awesome-orange, awesome-nephritis, awesome-concrete, awesome-darknight
% \colorlet{awesome}{awesome-red}
% Uncomment if you would like to specify your own color
\definecolor{awesome}{RGB}{204, 0, 0}

% Colors for text
% Uncomment if you would like to specify your own color
\colorlet{sectioncolor}{awesome}	% Names of sections
% \definecolor{darktext}			% Used in institution names
\colorlet{text}{darktext}			% Used in descriptions
% \definecolor{graytext}
% \definecolor{lighttext}
\colorlet{sectiondivider}{darktext}

% Set false if you don't want to highlight section with awesome color
\setbool{acvSectionColorHighlight}{false}

%-------------------------------------------------------------------------------
%	PERSONAL INFORMATION
%	Comment any of the lines below if they are not required
%-------------------------------------------------------------------------------
% Available options: circle|rectangle,edge/noedge,left/right
% \photo[rectangle,edge,right]{./examples/profile}
\name{William}{Boyles}
%\position{Software Engineer}
%\address{123 Sesame Street, New York, NY 11201}

%\mobile{(+1) 123-456-7890}
\email{williamboyles22900@gmail.com}
\homepage{wmboyles.net}
\github{wmboyles}
\linkedin{wmboyles}


%-------------------------------------------------------------------------------
\begin{document}
	
	% Print the header with above personal information
	% Give optional argument to change alignment(C: center, L: left, R: right)
	\makecvheader[C]
	
	
	%-------------------------------------------------------------------------------
	%	CV/RESUME CONTENT
	%-------------------------------------------------------------------------------
	\cvsection{Education}
	\begin{cventries}
		\cventry
		{B.S. Computer Science \& Mathematics}		% Degree
		{North Carolina State University} 			% Institution
		{Raleigh, NC} 								% Location
		{2018 - 2022} 								% Date(s)
		{
			\begin{cvitems}
				\item {Valedictorian, Summa Cum Laude, Phi Beta Kappa, Dean's List (8 semesters)}
				\item {University Honors Program, Computer Science Honors Program}
			\end{cvitems}
		}
	\end{cventries}

	\cvsection{Experience}
	\begin{cventries}
		\cventry
		{Software Engineer}	% Job title
		{Microsoft} 		% Organization
		{Redmond, WA} 		% Location
		{August 2022 -} 	% Date(s)
		{
			\begin{cvitems} % Description(s) of tasks/responsibilities
				\item {Created near-real-time Spark data processing pipelines that handle terabytes of pricing data per day}
				\item {Automated common interactions with production, reducing mitigation times and risk of unintended changes}
				\item {Automated deployment of cloud infrastructure to airgapped environments, minimizing the need for manual steps}
			\end{cvitems}
		}
	
		\cventry
		{Software Engineering Intern}	% Job title
		{Microsoft} 					% Organization
		{Redmond, WA} 					% Location
		{Summer 2021} 					% Date(s)
		{}
	
		\cventry
		{Cloud \& Cognitive Software Intern}	% Job title
		{IBM} 									% Organization
		{Durham, NC} 							% Location
		{Summer 2020} 							% Date(s)
		{}
	
		\cventry
		{Engineering Camp Counselor}	% Job title
		{Forsyth Country Day School} 	% Organization
		{Lewisville, NC} 				% Location
		{Summers 2016 - 2019} 			% Date(s)
		{}
		
		\cventry
		{Lifeguard} 			% Job title
		{Pool Professionals}	% Organization
		{Winston Salem, NC} 	% Location
		{Summers 2017 - 2019} 	% Date(s)
		{}
	\end{cventries}
	
	\cvsection{Skills}
	
	\begin{cvparagraph}
	
		
	\entrytitlestyle{Languages:}
		{ Python } \textbar 
		{ Scala } \textbar
		{ C\# } \textbar 
		{ SQL } \textbar
		{ Java } \textbar
		{ JavaScript } \textbar 
		{ \LaTeX } \textbar
		{ HMTL } \newline\newline
	\entrytitlestyle{Technologies:}
		{ Spark } \textbar
		{ Azure } \textbar
		{ Bootstrap } \textbar
		{ Git }
	\end{cvparagraph}
	
	\cvsection{Projects}
	\begin{cventries}
		\cvprojectentry
		{Lights Out} % Project Title
		{Android App} % Project Type
		{
			\begin{cvitems} % Descriptions
				\item {Based on 1990’s handheld electronic game, but has more features like dynamic board sizes}
				\item {Written in Java using Android Studio, released for free on Google Play Store}
			\end{cvitems}
		}
	
		\cvprojectentry
		{NCAA Bracket Prediction} % Project Title
		{Python Tool} % Project Type
		{
			\begin{cvitems} % Descriptions
				\item {Automatically scrapes the latest game data and outputs pre-filled bracket predictions as a \LaTeX-compiled PDF}
				\item {Python tool that implements PageRank, Elo, Bradley, and other custom ranking algorithms}
				\item {Algorithms consistently outperform average prediction tournament entries and my personal predictions}
			\end{cvitems}
		}
	
		\cvprojectentry
		{wmboyles.net} % Project Title
		{Personal Website} % Project Type
		{
			\begin{cvitems} % Descriptions
				\item {Personal domain containing resume, detailed project write-ups, and contact info}
				\item {Designed using Bootstrap and Jekyll, increasing mobile usability while minimizing code redundancy}
			\end{cvitems}
		}
	
		\cvprojectentry
		{Math Summaries} % Project Title
		{Open Source Math Textbooks} % Project Type
		{
			\begin{cvitems} % Descriptions
				\item {Summaries of college math courses written in \LaTeX ~and published as PDFs, allowing community contributions}
				\item {Committed to the public domain, providing educational information for free}
			\end{cvitems}
		}
	\end{cventries}

	\cvsection{Writing}
	\begin{cventries}
		\cvprojectentry
		{Most Clicks Problem in \textit{Lights Out}} % Project Title
		{Research Article} % Project Type
		{
			\begin{cvitems} % Descriptions
				\item {Introduced the Most Clicks Problem, and solves it for two specific instances}
			\end{cvitems}
		}
	
		\cvprojectentry
		{Resolution to Sutner's Conjecture} % Project Title
		{Research Article} % Project Type
		{
			\begin{cvitems} % Descriptions
				\item {Proves 30+ year old conjecture related to the Lights Out Game}
			\end{cvitems}
		}
	\end{cventries}
	
	
\end{document}


%-------------------------------------------------------------------------------
% CONFIGURATIONS
%-------------------------------------------------------------------------------
% A4 paper size by default, use 'letterpaper' for US letter
\documentclass[12pt, a4paper]{awesome-cv}

% Configure page margins with geometry
\geometry{left=0.75in, top=0.5in, right=0.75in, bottom=0.5in, footskip=0.5in}

% Color for highlights
% Awesome Colors: awesome-emerald, awesome-skyblue, awesome-red, awesome-pink, awesome-orange
%                 awesome-nephritis, awesome-concrete, awesome-darknight
% \colorlet{awesome}{awesome-red}
% Uncomment if you would like to specify your own color
\definecolor{awesome}{RGB}{204, 0, 0}

% Colors for text
% Uncomment if you would like to specify your own color
\colorlet{sectioncolor}{awesome}	% Names of sections
% \definecolor{darktext}			% Used in institution names
\colorlet{text}{darktext}			% Used in descriptions
% \definecolor{graytext}
% \definecolor{lighttext}
\colorlet{sectiondivider}{darktext}

% Set false if you don't want to highlight section with awesome color
\setbool{acvSectionColorHighlight}{false}

% If you would like to change the social information separator from a pipe (|) to something else
\renewcommand{\acvHeaderSocialSep}{\quad\textbar\quad}

%-------------------------------------------------------------------------------
%	PERSONAL INFORMATION
%	Comment any of the lines below if they are not required
%-------------------------------------------------------------------------------
% Available options: circle|rectangle,edge/noedge,left/right
% \photo[rectangle,edge,right]{./examples/profile}
\name{William}{Boyles}
%\position{Software Engineer}
%\address{123 Sesame Street, New York, NY 11201}

%\mobile{(+1) 123-456-7890}
\email{wmboyles@wmboyles.com}
%\dateofbirth{January 1st, 1970}
\homepage{wmboyles.com}
\github{wmboyles}
\linkedin{wmboyles}
% \gitlab{gitlab-id}
% \stackoverflow{SO-id}{SO-name}
% \twitter{@twit}
% \skype{skype-id}
% \reddit{reddit-id}
% \medium{madium-id}
% \kaggle{kaggle-id}
% \googlescholar{googlescholar-id}{name-to-display}
%% \firstname and \lastname will be used
% \googlescholar{googlescholar-id}{}
% \extrainfo{extra information}

%\quote{``Be the change that you want to see in the world."}


%-------------------------------------------------------------------------------
\begin{document}
	
	% Print the header with above personal information
	% Give optional argument to change alignment(C: center, L: left, R: right)
	\makecvheader[C]
	
	
	%-------------------------------------------------------------------------------
	%	CV/RESUME CONTENT
	%-------------------------------------------------------------------------------
	\cvsection{Education}
	\begin{cventries}
		\cventry
		{B.S. in Computer Science and Mathematics}	% Degree
		{North Carolina State University} 			% Institution
		{Raleigh, NC} 								% Location
		{2018 - 2022} 					% Date(s)
		{
			\begin{cvitems}
				\item {Valedictorian, Summa Cum Laude, Phi Beta Kappa, Dean's List (8 semesters)}
				\item {University Honors Program, Computer Science Honors Program}
			\end{cvitems}
		}
	\end{cventries}

	\cvsection{Experience}
	\begin{cventries}
		\cventry
		{Software Engineering Intern} % Job title
		{Microsoft} % Organization
		{Redmond, WA} % Location
		{Summer 2021} % Date(s)
		{
			\begin{cvitems} % Description(s) of tasks/responsibilities
				\item {Created Azure resources to monitor critical infrastructure for failures, improving response times}
				\item {Built monitoring tools in Azure for purchase infrastructure, ensuring government compliance}
				\item {Deployed solutions to production and airgapped government clouds}
			\end{cvitems}
		}
	
		\cventry
		{Cloud \& Cognitive Software Intern} % Job title
		{IBM} % Organization
		{Durham, NC} % Location
		{Summer 2020} % Date(s)
		{
			\begin{cvitems} % Description(s) of tasks/responsibilities
				\item {Created Python tool to visualize cloud outages and identify root causes in real-time, driving response improvements}
				\item {Developed Python Slack bot to provide actionable, on-demand data to outage responders}
				\item {Overhauled data pipeline via a technical redesign, increasing speeds by up to 5900\%}
			\end{cvitems}
		}
	
		\cventry
		{Engineering Camp Counselor} % Job title
		{Forsyth Country Day School} % Organization
		{Lewisville, NC} % Location
		{Summers 2016 - 2019} % Date(s)
		{}
		
		\cventry
		{Lifeguard} % Job title
		{Pool Professionals} % Organization
		{Winston Salem, NC} % Location
		{Summers 2017 - 2019} % Date(s)
		{}
	\end{cventries}
	
	\cvsection{Projects}
	\begin{cventries}
		\cvprojectentry
		{Lights Out} % Project Title
		{Android Mobile App} % Project Type
		{
			\begin{cvitems} % Descriptions
				\item {Based on 1990’s handheld electronic game, but has more features like dynamic board sizes}
				\item {Written in Java using Android Studio}
				\item {Released for free on Google Play Store for all Android devices}
			\end{cvitems}
		}
	
		\cvprojectentry
		{NCAA Bracket Prediction} % Project Title
		{Python Tool} % Project Type
		{
			\begin{cvitems} % Descriptions
				\item {Automatically scrapes the latest game data and outputs predictions as a \LaTeX-compiled PDF}
				\item {Python tool that implements PageRank, Elo, Bradley, and other custom ranking algorithms}
				\item {Algorithms consistently outperform average prediction tournament entries and my personal predictions}
			\end{cvitems}
		}
	
		\cvprojectentry
		{wmboyles.com} % Project Title
		{Personal Website} % Project Type
		{
			\begin{cvitems} % Descriptions
				\item {Personal domain containing resume, detailed project write-ups, and contact info}
				\item {Overhauled design to use Bootstrap, increasing mobile usability}
				\item {Built using Jekyll, minimizing code redundancy}
			\end{cvitems}
		}
	
		\cvprojectentry
		{Math Summaries} % Project Title
		{Open-Source Math Textbooks} % Project Type
		{
			\begin{cvitems} % Descriptions
				\item {Summaries of math courses I took in college, including multivariable calculus and differential equations}
				\item {Written in \LaTeX ~and published as PDFs, allowing open source contributions}
				\item {Committed to the public domain, allowing users to always access information for free}
			\end{cvitems}
		}
	\end{cventries}
	
	
\end{document}

